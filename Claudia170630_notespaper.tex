%% LyX 2.2.1 created this file.  For more info, see http://www.lyx.org/.
%% Do not edit unless you really know what you are doing.
\documentclass[english]{paper}
\usepackage{mathptmx}
\usepackage[T1]{fontenc}
\usepackage[latin9]{inputenc}
\usepackage{geometry}
\geometry{verbose}
\setcounter{secnumdepth}{2}
\setcounter{tocdepth}{2}
\usepackage{amsmath}
\usepackage{amssymb}
\usepackage{babel}
\begin{document}

\subsubsection{Introduction}

A fundamental problem in brain research consists in the high variability
between individual brains. Especially when it comes to disease diagnosis,
it is often hard to find a unique measurable quantity with non-overlapping
value distributions  such that healthy and diseased subjects can
be unmistakably distinguished. As a consequence, a better disease
classification should be reached, if a variety of measured properties
are taking into account, even if their value distributions strongly
overlap. The resulting high dimensionality space spanned by the different
measured properties is difficult to handle unless classifier such
SVM and ... are applied. But classifiers usually need a large sample
size for training to prevent underfitting, which often cannot be provided.

Here we provide a statistical framework that handles high dimensional
data, deals with small data sets and estimates in how far more data
would improve prediction. In addition, it can be easily updated by
data that is handed in later on and extracts the most informative
parameters. Furthermore, the underlying probability model, the so
called model by sufficiency, does not require any predefined statistical
model, because it is uniquely determined by the assumption that the
means and covariances of the measured properties are sufficient to
make future predictions. 

In this work we apply this statistical framework to a particular data
set of resting-state functional magnet resonance imaging (rfMRI) data
of schizophrenic (SZ) and control (C) individuals. The data was freely
provided by Schizonet a virtual database for public schizophrenia
neuroimaging data. Since SZ is a disease that affects the entire brain,
as it is explained in more detail below, we expect that analyzing
functional connectivity results in high dimensional data with overlapping
properties distributions well suited for our analysis.

SZ is a psychiatric disorder that comprises various symptoms that
are categorized into positive (e.g. hallucinations), negative (e.g.
loss of motivation) and cognitive (e.g. memory impairment) disease
patterns. A common disease cause for all these widespread symptoms
is not found yet. However, many studies found profound changes in
macroscopic brain structures, e.g. a shrinkage of whole brain and
ventricular volume, reduced gray matter in frontal, temporal cortex
and Thalamus, and changes in white matter volume in frontal and temporal
cortex.\cite{Shenton2010,Wright2009,Wright2010} Since both gray
matter loss and white matter changes are found, it is reasonable to
conclude that not only the intrinsic activity of single areas is modified
but also the interplay of different brain areas in particular in frontal
and temporal cortex. It is even argued that these alterations in long
range connectivity are responsible for a range of disease symptoms
that are not attributed to single areas.\cite{Friston1995} Taking
this disconnect hypothesis as starting point, research more and more
focuses on functional connectivity. 

In general, functional connectivity is measured either by asking the
subject to fulfill a certain task or at rest, instructing the subject
to think about nothing specific but not fall asleep. At rest, changes
in functional connectivity of different subnetworks are reported in
SZ. For example, both increased and decreased functional connectivity
is found in the default mode network (DMN), but the hyperactivity
seems to be reported more often.\cite{Hu2017} Moreover widespread
connectivity changes in the dorsal attention network (DAN) and the
executive control network (ECN) are detected.\cite{Woodward2011,Yu2012}

In order to account for all these possible changes in functional connectivities
across the different networks, we construct graphs with edges representing
the functional connectivity between two cortex areas. Comparing the
distribution of individual edge weights across the SZ subjects with
the one of the controls, we find highly overlapping distributions.
But despite this high overlap we achieve a (very) good separability
of the two groups by applying the model by sufficiency framework.

\subsubsection{Data acquisition}

We requested data of Schizophrenic patients and control individuals
from Schizconnect (http://schizconnect.org), a virtual database for
public schizophrenia neuroimaging data. In our request we asked for
resting state T2{*}-weighted functional and T1-weighted structural
magnet resonance images (MRI) from individuals participating in the
COBRE study either with no known disorder or diagnosed as schizophrenic
according to the Diagnostic and Statistical Manual of Mental Disorders
(DSM) IV, excluding schizoaffective disorders. The resultant provided
data set comprised 91 control individuals and 74 schizophrenic patients
, whose voluntary and informed participation in the COBRE study was
ensured by the institutional guidelines at the University of New Mexico
Human Research Protections Office. A detailed description on the
exact experimental design and the MRI scanning is provided by (citation:
Cetin at all).

\subsubsection{Data preprocessing}

Preprocessing of the rfMRI images is carried out by FSL (Smith et
al., 2004; Jenkinson et al., 2012) including the following steps:
removal of the first 10 images; motion correction (MCFLIRT); spacial
filtering with a 4mm FWHM Gaussian kernel; temporal high-pass filtering
with a cut-off frequency of 0.2Hz; white matter and CSF regression
(fsl\_regfilt).

\subsubsection*{Graph construction}

For each subject we register the rfMRI image first to structural space
(FSL) and then to the MNI standard space (Ants). Regions of the resulting
functional image in standard space are extracted such that they match
the 94 regions identified by the Oxford lateral cortical atlas (with
a probability above 50 percents). The resulting regions of interest
form the nodes of the individual graphs. Edge weights are measured
based on the (absolute) value of the Pearson correlation coefficient
calculated the mean activity of the two nodes considered. Thus, we
end up with a weighted, undirected, fully connected graph of 94 nodes
for each patient. 

\subsubsection*{Selection of graph weights}

Because we want to avoid computational overload we pe-select a small
subset of all possible 4371 edges. The pre-selection was done such
that we considered only ten edge weights with the highest differences
in the means between the two populations.

\subsubsection{}

\subsection*{Discussion}
\begin{itemize}
\item medication
\item graph weights as diagnosis tools 
\item classifier often used, entire data
\end{itemize}

\bibliographystyle{brainplain}
\addcontentsline{toc}{section}{\refname}\bibliography{brain}

\end{document}
