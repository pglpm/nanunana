%\pdfoutput=1
%% luca161203-question_ZR_sufficient.tex --- 
%% 
%% Author: PGL  Porta Mana
%% Created: 2016-12-03T14:02:45+0100
%% Last-Updated: 2016-12-04T23:30:29+0100
%%%%%%%%%%%%%%%%%%%%%%%%%%%%%%%%%%%%%%%%%%%%%%%%%%%%%%%%%%%%%%%%%%%%%%
%Report-no: ***
\newcommand*{\memfontfamily}{zplx}
\newcommand*{\memfontpack}{newpxtext}
\documentclass[10pt,%extrafontsizes,%
%dvips,% comment this for arXiv
onecolumn,oneside,a5paper,article,frenchb,italian,german,swedish,latin,british%
%,draft%***
%,final%
]{memoir}
\newif\ifnotnotes
\notnotesfalse % true = for publication, false = personal notes
\newcommand*{\pdftitle}{A question on a model via sufficiency}
\newcommand*{\headtitle}{\pdftitle}
\newcommand*{\pdfauthor}{P.G.L.  Porta Mana}
\newcommand*{\headauthor}{\ifnotnotes Porta Mana%
\else\autanet\ Luca\fi}
\newcommand*{\reporthead}{}

%%% Call to the various packages, most of them in bringhurst3.sty %%%
%\usepackage[25-31,files]{pagesel}
\usepackage[%nomicrotype,%
%nosum,%
%nomathpazo% loads mathpazo - Palatino text & math
%,noosf% no old-style figures
%,fakesc,%,
%noeugreek,% no greek letters in Euler font
%,nooptima% no Zapf's Optima font as sans serif
%,noitsans% no math sans serif in italic
%,nolmodern% use Latin Modern instead of CM
%,noeucal%
%,commadec% print a comma as dec separator when . is used
%nodejavutt,% DejaVu Mono as typewriter font
%,bb% blackboard letters for naturals, integers, reals, etc.
datetime]{bringhurst3}

%%%% Paper's details %%%%
\title{\pdftitle%***
}
\author{\ifnotnotes% 
%\hspace*{\stretch{1}}\protect\makebox[0pt][c]%
%{\firstname{***}\ \surname{***}}%
%\hspace*{\stretch{1}}\protect\makebox[0pt][c]%
{\firstname{P.G.L.}\ \surname{Porta\,Mana}}%
%\hspace*{\stretch{1}}%
\\[2\jot]%
\affiliation{INM-6, Forschungszentrum Jülich, Germany}
% \\[2\jot]%
\else Luca\fi
\quad
\epost{\email{pglpm0}{gmail.com}}%
}

\date{Draft of \mydate\today\ (first drafted 3 December 2016)}

%@@@@@@@@@@@@@@@@@@@@@@@ new commands @@@@@@@@@@@@@@@@@@@@@
\definecolor{notecolour}{RGB}{68,170,153}
\newcommand*{\mynote}[1]{ {\color{notecolour}\maltese\ #1}}
%\DeclareMathOperator{\artanh}{artanh}
%\DeclareMathOperator{\nabnab}{\bm{\nabla\nabla}}
%\newcommand*{\nabnab}{\mathord{\bm{\nabla}\otimes\bm{\nabla}}}
\newcommand*{\half}{\tfrac{1}{2}}
\newcommand*{\te}{}
\newcommand*{\co}{}
\newcommand*{\yxn}{X}
\newcommand*{\yI}{I}
\newcommand*{\yS}{S}
\newcommand*{\data}{\textrm{data}}
\newcommand*{\yf}{f}
\newcommand*{\yh}{q}
\newcommand*{\bx}{\bm{x}}
\newcommand*{\ya}{\textrm{A}}
\newcommand*{\yb}{\textrm{B}}
\newcommand*{\yna}{n_{\ya}}
\newcommand*{\ynb}{n_{\yb}}
\newcommand*{\ynh}{n_{\yh}}
\newcommand*{\yxa}{\overline{\bx}_{\ya}}
\newcommand*{\yxb}{\overline{\bx}_{\yb}}
\newcommand*{\yxh}{\overline{\bx}_{\yh}}
\newcommand*{\yxxa}{\overline{\bx\te\bx}_{\ya}}
\newcommand*{\yxxb}{\overline{\bx\te\bx}_{\yb}}
\newcommand*{\yxxh}{\overline{\bx\te\bx}_{\yh}}
\newcommand*{\yla}{\alpha_{\ya}}
\newcommand*{\ylb}{\alpha_{\yb}}
\newcommand*{\yma}{\bm{\beta}_{\ya}}
\newcommand*{\ymb}{\bm{\beta}_{\yb}}
\newcommand*{\ymma}{\bm{\varGamma}_{\ya}}
\newcommand*{\ymmb}{\bm{\varGamma}_{\yb}}
\newcommand*{\yfa}{\nu_{\ya}}
\newcommand*{\yfb}{\nu_{\yb}}
\newcommand*{\yff}{\nu}
\newcommand*{\yca}{\bm{m}_{\ya}}
\newcommand*{\ycb}{\bm{m}_{\yb}}
\newcommand*{\yssa}{\bm{V}_{\ya}}
\newcommand*{\yssb}{\bm{V}_{\yb}}
\newcommand*{\yth}{\theta}
\newcommand*{\yN}{N}
%@@@@@@@@@@@@@@@@@@@@@@@ new commands end @@@@@@@@@@@@@@@@@

\firmlists
\begin{document}
\captiondelim{\quad}\captionnamefont{\footnotesize}\captiontitlefont{\footnotesize}
\selectlanguage{british}\frenchspacing

%%% Title and abstract %%%
\maketitle
\ifnotnotes\myabstract
\begin{abstract}\labelsep 0pt%
\noindent ***
\par%\\[\jot]
\noindent\pacs{***}\qquad\msc{***}
\end{abstract}\fi

\selectlanguage{british}\frenchspacing
%\asudedication{to ***}
%\setlength{\epigraphwidth}{.7\columnwidth}
%\setlength{\epigraphrule}{0pt}
%\epigraph{}

\addsubsec{Notation}

Boldface letters: Latin ones are contravariant quantities: lowercase
are vectors and uppercase are rank-2 tensors; Greek ones are covariant
quantities: lowercase are covectors and uppercase are rank-2 tensors.
Juxtaposition of two quantities indicates tensor product if they are of the same
transformation kind and contraction if they are of opposite kind:
$\bx\te\bx\equiv\bx\otimes\bx$ is rank-2 and
$\bm{\alpha}\co\bx\equiv\bm{\alpha}\cdot\bx$ is a scalar. \iffalse Full contraction
is indicated by a colon: $\bm{\varGamma}\con\bm{G}$ is a scalar.\fi

Medium letters denote scalars and combinations of quantities of different
type, depending on the context


\section{Context}
\label{sec:context}

I have a keen interest in the method of \emph{modelling via sufficiency},
which in rough terms works as follows. We have the outcomes $\set{X_i}$
of some observations made with the same set-up. Let's call these outcomes
our \enquote{data}. We want to predict the yet unknown outcome $\yxn$ of one
more observation, same set-up, given our data and some background knowledge
$\yI$; that is, we want to give for each $\yxn$ a numerical value to
\begin{equation}
  \label{eq:P_new_outcome}
  \pf(\yxn \| \data, \yI).
\end{equation}

Now suppose we believe or just entertain the hypothesis that for this
prediction we only need to know a function -- an \enquote{aspect}, or
\enquote{feature} -- of the data $\yf(\data)$; the rest of the data is
irrelevant. That is, calling our hypothesis $\yS$,
\begin{equation}
  \label{eq:P_new_outcome_sufficiency}
\forall\yxn,\quad  \pf[\yxn \| \data, \yf(\data), \yS, \yI] =
  \pf[\yxn \| \yf(\data), \yS, \yI].
\end{equation}
An example is $\yf(\set{X_i})=X_1+X_2+\dotsb$ if the outcomes are numbers.
I call $\yS$ a hypothesis about sufficiency; $\yf$ is called
\emph{sufficient statistics}.

A consequence of the hypothesis is also that
\begin{equation}
  \label{eq:P_different_data_sufficiency}
  \tag*{(\ref{eq:P_new_outcome_sufficiency}')}
\forall\data,\data',\quad  \pf(\data \| \yS, \yI) = \pf(\data' \| \yS, \yI)
\;  \text{if} \; \yf(\data)=\yf(\data'),
\end{equation}
that is, data with the same sufficient statistics must have the same
probability.

The method of modelling via sufficiency says that our hypothesis then leads
to a \emph{unique} expression for $\pf(\yxn \| \data, \yS, \yI)$, given our
prior guesses about the limit values in the absence of data. In other
words, our hypothesis leads to a specific value for this probability.

This method is based on mathematical theorems reviewed in several works
\cites[\eg:][]{dawid2013,diaconis1992}[\sect~4.5]{bernardoetal1994_r2000}{kallenberg2005,lauritzen1982_r1988,diaconisetal1981}.
These works also show the unique expression for our probability for
particular choices of outcome spaces and sufficient statistics. In many
important cases the resulting expression is an integral containing
distributions of the exponential family, for example the normal, exponential,
binomial, or multinomial ones.


\section{Question}
\label{sec:question}

\subsection{Two hypotheses}
\label{sec:two_hypotheses}

I am interested in a particular kind of outcomes: each outcome is a pair of
a \enquote{quality} $\yh$ and an \enquote{indicator} $\bx$; that is,
$X_i=(\yh_i,\bx_i)$. The quality is one from of a discrete set, for example
$\yh \in \set{\textrm{green}, \textrm{red}, \textrm{blue}}$, or
$\yh \in \set{\textrm{healthy}, \textrm{unhealthy}}$; the indicator is a
real-valued vector. The space of outcomes is therefore homeomorphic to
$\ZZ_n\times\RR^m$ for some $n,m$. Let's say for definiteness that there
are two qualities called $\ya$ and $\yb$.

First let's define the number of outcomes of each quality, $\yna,\ynb$, and
the averages of the first and second powers of the indicators for each
quality, $\yxa,\yxb,\yxxa, \yxxb$:
\begin{equation}
  \label{eq:suff_stat_1}
  \begin{aligned}
      \yna&\defd \tsum_i\delt(\yh_i=\ya),
&
      \ynb&\defd \tsum_i\delt(\yh_i=\yb),
\\
\yxa&\defd \tsum_i \bx_i\delt(\yh_i=\ya)/\yna,
&
\yxb&\defd \tsum_i \bx_i\delt(\yh_i=\yb)/\ynb,
\\
\yxxa&\defd \tsum_i {\bx_i}\te\bx_i\delt(\yh_i=\ya)/\yna,
&
      \yxxb&\defd \tsum_i {\bx_i}\te\bx_i\delt(\yh_i=\yb)/\ynb.
  \end{aligned}
\end{equation}
  
I would like to know the form of the distribution
$\pf[\yxn \| \yf(\data), \yS, \yI]$ for two choices of sufficient
statistics among the quantities above, that is, two different hypotheses
about sufficiency $\yS$ given data $\set{(\yh_i,\bx_i)}$:\defaultlists
\begin{description}
\item[$\yS_1$:] The sufficient statistics $\yf_1(\data)$ are: the number of
  outcomes of each quality and the averages of the indicators for each
  quality. In formulae, $\yf_1(\data)\defd\set{\yna,\ynb,\yxa,\yxb,\yxxa,\yxxb}$.
  
  This sufficiency condition is mathematically expressed by
  \begin{equation}
    \label{eq:P_suff_stat_S1}
    \pf[(\yh,\bx) \| \data, \yS_1, \yI] =
    \pf[(\yh,\bx) \| \yna,\ynb,\yxa,\yxb,\yxxa,\yxxb, \yS_1, \yI].
  \end{equation}

  There are other equivalent ways of specifying this statistics, for
  example by giving the total number of outcomes, the number of those of
  quality $\ya$, and the totals of the indicators for each quality:
  $\set{\yna+\ynb, \yna, \yna\yxa, \ynb\yxb}$.
  
\item[$\yS_2$:] This hypothesis combines sufficiency on two different
  levels: the probability for the qualities $\set{\yh_i}$ is exchangeable,
  and that of the indicators given the qualities is \enquote{unrestrictedly
  exchangeable} \cite[\sect~4.6.2]{bernardoetal1994_r2000}. In terms of
sufficient statistics: the total number of each quality in the data is a
sufficient statistics for $\yh$:
\begin{subequations}\label{eq:P_suff_stat_S2}
  \begin{equation}
    \label{eq:suff_stat_h}
    \pf(\yh \| \data, \yS_2,\yI) = \pf(\yh \| \set{\yh_i}, \yS_2,\yI)
    = \pf(\yh \| \yna, \ynb, \yS_2,\yI); 
  \end{equation}
  and the total of the first and second powers of the indicator related to
  a particular quality is a sufficient statistics for that indicator, if
  the qualities are known:
  \begin{equation}
    \label{eq:suff_stat_x}
    \pf(\bx \| \yh, \data, \yS_2,\yI)
    = \pf(\bx \| \yh, \ynh, \yxh, \yxxh, \yS_2,\yI).
  \end{equation}
\end{subequations}
\end{description}\firmlists

The two conditions~\eqref{eq:P_suff_stat_S2} together imply the same
condition~\eqref{eq:P_suff_stat_S1} of the first hypothesis, but not vice
versa. So hypothesis $\yS_2$ seems to be a special case of hypothesis
$\yS_1$.

\subsection{Integral form}
\label{sec:integral_form}

We calculate the joint probability for an arbitrary sequence
$\set{(\yh_i,\bx_i)}$ under the two hypotheses. From this all other
  probabilities can be calculated.

\bigskip

$\yS_1$. Following Bernardo \amp\ Smith's discussion on sufficiency and the
exponential family \cite[\sect~4.5.3]{bernardoetal1994_r2000} I assume that
hypothesis $\yS_1$ leads to the following form for the joint probability:
\begin{multline}
  \label{eq:P_integral_exp_S1_initial}
  \pf[\set{(\yh_i,\bx_i)} \| \yS_1, \yI]
  ={}\\
  \qquad\!\begin{aligned}[b]
    \int&\prod_i\Bigl\{\tfrac{1}{Z(\yth')} \exp[\!\begin{aligned}[t]
      &\yla \delt(\yh_i=\ya) +\ylb \delt(\yh_i=\yb) + {}\\
      &\yma \bx_i\delt(\yh_i=\ya) +\ymb \bx_i\delt(\yh_i=\yb) +{}\\
      &\bx_i\ymma\bx_i \delt(\yh_i=\ya) +\bx_i\ymmb\bx_i \delt(\yh_i=\yb)
      ] \Bigr\}\times{}
    \end{aligned}\\
    &\pf(\yth' \cond \yS_1,\yI)\, \di\yth'
  \end{aligned}
\end{multline}
with $\yth'=(\yla,\ylb,\yma,\ymb,\ymma,\ymmb)$ and $Z(\yth')$ a
normalization factor for the exponential. Either of the terms $\yla$ or
$\ylb$ is actually redundant and is eliminated via a Dirac delta in the
parameter prior $\pf(\yth' \cond \yS_1,\yI)$. Compare Definition~4.11 and
Proposition~4.13 (p.~201) \cite[\sect~4.5.3]{bernardoetal1994_r2000}. The
mathematical type of the integration variables should be clear from the
context.

The reason for the prime in $\yth'$ is that we now change to a more
convenient set of parameters $\yth=(\yfa,\yfb,\yca,\yssa,\ycb,\yssb)$:
\begin{equation}
  \label{eq:new_params}
  \begin{split}
    \yla &= \ln\frac{\yfa}{\yfa+\yfb} - \half\yca{\yssa}^{-1}\yca
    - \ln\det(2\pu\yssa),\\
    \yma &= \yca{\yssa}^{-1},\\
    \ymma &= -\half{\yssa}^{-1}
    \quad\text{(inversion exchanges co- and contravariance),}
  \end{split}
\end{equation}
and similarly for $(\ylb, \ymb, \ymmb)$. Combining the \textsc{Jakob}ian
determinant of the transformation into the parameter prior, and denoting
the normal distribution with $\yN$, the joint probability now takes the
form
\begin{equation}
  \label{eq:P_integral_exp_S1_final}
  \begin{split}
    \pf[\set{(\yh_i,\bx_i)} \| \yS_1, \yI]&=\int \!\begin{aligned}[t]
      &\prod_i\Bigl[ \!\begin{aligned}[t]
        &\yfa^{\delt(\yh_i=\ya)} \yN(\bx_i \| \yca,\yssa)^{\delt(\yh_i=\ya)}\times{}\\
        &\yfb^{\delt(\yh_i=\yb)} \yN(\bx_i
        \|\ycb,\yssb)^{\delt(\yh_i=\yb)} \Bigr] \times{}
      \end{aligned}\\
      &\pf(\yfa,\yfb,\yca,\yssa,\ycb,\yssb\cond \yS_1,\yI)\, \di\yth
    \end{aligned}
    \\
    &=\int
    \!\begin{aligned}[t]
      &\yfa^{\yna} \prod_i^{\yh_i=\ya}\yN(\bx_i \| \yca,\yssa)\times{}\\
      &\yfb^{\ynb} \prod_i^{\yh_i=\yb}\yN(\bx_i \| \ycb,\yssb)\times{}\\
      &\pf(\yfa,\yfb,\yca,\yssa,\ycb,\yssb\cond \yS_1,\yI)\, \di\yth,
    \end{aligned}
  \end{split}
\end{equation}
with a Dirac delta $\delt(\yfa+\yfb-1)$ in the parameter prior. The new
normalization factor turns out to be unity. This expression is a mixture of
products of Bernoulli distributions for the qualities and normal
distributions for the indicators. The normals are of two kinds depending on
either quality corresponding to each indicator.

\bigskip

$\yS_2$. From Bernardo \amp\ Smith's discussion on exchangeability
\cite[\sect~4.3]{bernardoetal1994_r2000}, exchangeability for the
probability of the qualities $\set{\yh_i}$, \eqn~\eqref{eq:suff_stat_h}
leads to
\begin{equation}
  \label{eq:P_integral_S2_h}
  \begin{split}
    \pf(\set{\yh_i} \| \yS_2, \yI) &= \int \prod_i\Bigl[\yfa^{\delt(\yh_i=\ya)}
    \yfb^{\delt(\yh_i=\yb)}\Bigr] \,\pf(\yfa,\yfb\cond \yS_1,\yI)
    \,\di\yfa\,\di\yfb\\
    &=\int\yfa^{\yna} \yfb^{\ynb}\,\pf(\yfa,\yfb\cond \yS_1,\yI)
    \,\di\yfa\,\di\yfb,
  \end{split}
\end{equation}
with the usual implicit Dirac delta.

From the discussion on unrestricted exchangeability
\cite[\sect~4.6.2]{bernardoetal1994_r2000}, the probability of the
indicators $\set{\bx_i}$ given the corresponding qualities $\set{\yh_i}$ is
\begin{equation}
  \label{eq:P_integral_S2_x}
  \!\begin{aligned}
    \pf(\set{\bx_i} \cond \set{\yh_i}, \yS_2, \yI) = {}&\int
    \!\begin{aligned}[t]
    &\prod_i^{\yh_i=\ya}\yN(\bx_i \| \yca,\yssa)\times{}\\
    &\pf(\yca,\yssa \| \yS_2,\yI)\,\di\yca\,\di\yssa\times{}
  \end{aligned}\\
&\int\!\begin{aligned}[t]
    &\prod_i^{\yh_i=\yb}\yN(\bx_i \| \ycb,\yssb)\times{}\\
    &\pf(\ycb,\yssb \| \yS_2,\yI)\,\di\ycb\,\di\yssb.
  \end{aligned}
  \end{aligned}
\end{equation}

Multiplying the above two probabilities we find
\begin{equation}
  \label{eq:P_integral_exp_S2_final}
  \!\begin{aligned}
    \pf[\set{(\yh_i,\bx_i)} \| \yS_2, \yI]=\int
    &\yfa^{\yna} \prod_i^{\yh_i=\ya}\yN(\bx_i \| \yca,\yssa)\times{}\\
    &\yfb^{\ynb} \prod_i^{\yh_i=\yb}\yN(\bx_i \| \ycb,\yssb)\times{}\\
    &\pf(\yfa,\yfb\cond \yS_2,\yI)\times{}\\
    &\pf(\yca,\yssa\cond \yS_2,\yI)\, 
    \pf(\ycb,\yssb\cond \yS_2,\yI)\, \di\yth.
  \end{aligned}
\end{equation}

\subsection{Comparison}
\label{sec:comparison}

The final integral forms, \eqns~\eqref{eq:P_integral_exp_S1_final}
and~\eqref{eq:P_integral_exp_S2_final}, of the joint probabilities under
the two hypotheses, $\pf[\set{(\yh_i,\bx_i)} \| \yS_1, \yI]$ and
$\pf[\set{(\yh_i,\bx_i)} \| \yS_2, \yI]$, shows that they only differ in
the parameter prior distribution, which factorizes in the second hypothesis
$\yS_2$. This confirms our previous remark that $\yS_2$ seems to be a
special case of $\yS_1$.

The factorized form of the parameter prior in the second hypothesis is
connected to the fact that data about indicators for quality $\yb$ are
irrelevant for prediction of the indicator for quality $\ya$, and vice
versa. This is clear by constructing that conditional probability from
\eqn~\eqref{eq:P_integral_exp_S2_final}. This irrelevance is generally not
true under the first hypothesis.

The use of a flat parameter prior (and factorizing boundaries)
under the first hypothesis erases its difference from the second one.

\subsection{Questions}
\label{sec:questions_final}

My questions are these:

\begin{enumerate}
\item It is easy to show that the integral forms of
  \eqns~\eqref{eq:P_integral_exp_S1_final},
  \eqref{eq:P_integral_exp_S2_final} for the joint probabilities imply the
  assumptions \eqref{eq:P_suff_stat_S1}, \eqref{eq:P_suff_stat_S2} of the
  corresponding hypotheses. Is the converse true?

\item I assumed that the exponential form in the integral also holds for
  variables of mixed, discrete-continuous character. Is this true? Possibly
  Barndorff-Nielsen \citey{barndorffnielsen1978_r2014} has the answer?

\item A flat parameter prior makes hypothesis $\yS_1$ and $\yS_2$
  equivalent. Does this make sense? Why does a flat prior introduce
  independence between indicators of different qualities for example?

\item Are the calculations in this note correct?
\end{enumerate}
  
\iffalse a
\begin{equation}
  \abs{abc}=\abs{abc}+(\abs{abc})
\end{equation}
\begin{equation}
  \abs{abc}={}
  \!\begin{aligned}[t]
    &\abs{abc}+(
    \!\begin{aligned}[t]
      &\abs{abc}+{}\\ &\abs{abc}) \times{}
    \end{aligned}\\
&\abs{abc}
  \end{aligned}
\end{equation}
\begin{equation}
  \!\begin{aligned}
    \abs{abc}={}&
    \!\begin{aligned}[t]
      \abs{abc}+(&\abs{abc}+{}\\ &\abs{abc}) \times{}
    \end{aligned}\\
&\abs{abc}
  \end{aligned}
\end{equation}\fi



\iffalse
\begin{figure}[!b]
\centering
\includegraphics[width=0.4\columnwidth]{***}%
\caption{***}
\label{***}
\end{figure}
\fi

\ifnotnotes
\begin{acknowledgements}
  Many thanks to Mari \amp\ Miri for continuous encouragement and affection,
  to Buster for filling life with awe and inspiration, and to the
  developers and maintainers of \LaTeX, Emacs, AUC\TeX, MiK\TeX, arXiv,
  biorXiv, PhilSci, Hal archives, Python, Inkscape, Sci-Hub for making a
  free and unfiltered scientific exchange possible.
%\rotatebox{15}{P}\rotatebox{5}{I}\rotatebox{-10}{P}\rotatebox{10}{\reflectbox{P}}\rotatebox{-5}{O}.
\sourceatright{\autanet}
\end{acknowledgements}
\fi

%\appendixpage
%\appendix

%%%%%%%%%%%%%%% BIB %%%%%%%%%%%%%%%

\defbibnote{postnote}{\small\par\medskip\noindent{\footnotesize% Note:
\arxivp \mparcp \philscip \biorxivp}%
}

\newcommand{\citein}[2][]{\textnormal{\textcite[#1]{#2}}%\addtocategory{extras}{#2}
}
\newcommand{\citebi}[2][]{ref.\ \citep[#1]{#2}%\addtocategory{extras}{#2}
}

\printbibliography[postnote=postnote]


\end{document}
---------- cut text ----------------


%%% Local Variables: 
%%% mode: LaTeX
%%% TeX-PDF-mode: t
%%% TeX-master: t
%%% End: 
